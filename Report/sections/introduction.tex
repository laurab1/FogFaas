\documentclass[../DraftNotes.tex]{subfiles}

\begin{document}

The main objective of the project is to provide an extension of SecFog which solves the Components Deployment Problem adding some functionalities to meet the Faas paradigm, i.e. providing a way to deploy services on nodes of an infrastructure in such a way that some security requirements are met. In this setting, services are intended to be compositions of functions: the main point is thus to effectively and securely deploy functions on the given infrastructure. \\
To ensure a high level of security, we provide a type system which takes into account several kinds of implicit flows, togheter with further checks on the nodes. A labelling predicate must be defined by the user, in such a way that different security constraints can be easily defined. \\
Furhtermore, labels are intended to be ordered, so that we get a (semi)lattice which can be exploited to compute the security level of functions and, subsequently, of services. \\
As we model a Fog environment, we also provide constructs to allow communication between services and operations on resources, which might be not only files but also sensors or actuators. Moreover, we model triggers which allow clients to invoke a given service. \\
In the following, we cover the design choices and we provide the language syntax and semantics, along with some examples and use cases.

\end{document}