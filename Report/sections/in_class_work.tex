\documentclass[../DraftNotes.tex]{subfiles}

\begin{document}

The in class work mainly focused on the deployment of functions. As said, the main functionality provided by FogFaas is the secure deployment of functions on an infrastructure: to get this, we provide a \emph{placeApp} predicate. As an app is a list of services, the predicate call internally the \emph{placeService} predicate, which in turn calls the \emph{placeFunctions} one. Here (parallel or sequential) functions are deployed, according to the resources and the security guarantees of the nodes.
The codebase also embed the trust model of SecFog and can be customized with different trust models. Except for this part, the basic functionalities of FogFaas don't require a probabilistic reasoning (while, as we'll see in the following, probability plays a key role in modelling communication). \\
In this preliminary work, we had a quite simple type system and few language constructs. The language has been later extended with several instructions, as well as the type system which provides some new rules.

\end{document}