\documentclass[../DraftNotes.tex]{subfiles}

\begin{document}

We consider the problem of representing synchronous communications between services in a fog environment. In this setting, a deployment of the application is possible if the nodes hosting two communicating services are actually connected by a physical link which meet some security requirements as, for istance, some constraints on the geographical location of the traversed nodes. \\
In order to effectively represent communicating services, then, we have to represent the physical links and to provide a language construct for synchronous communications.
\smallskip
To tackle the problem, we opted for representing physical links as bidirectional links. This adds a level of complexity in finding a solution to the routing problem, as we have to solve it on a non-directed, cyclic graph, but it seemed a reasonable representation of a physical network. \\
However, as said, we are representing synchronous communications between services, thus we provide a single language construct \emph{send(Args, Service)} to this end. Furthermore, since synchronous calls can lead to starvation, we assume the instruction to be provided with a timeout, hence having two cases:
\begin{itemize}
	\item the send operation succeeds: then the sent value is stored by the receiving service;
	\item the send operation fails for exceeding timeout.
\end{itemize}
The possibility of failure is taken into account exploiting probability. Thus the user can provide the probability of timeout exceeding using the \emph{responseTime} predicate, which in turn is used inside the \emph{ctx} predicate. If the communication succeeds, then the security context of the receiving service is updated according to the security level of the received value, otherwise it stays unchanged. 
\end{document}