\documentclass[../DraftNotes.tex]{subfiles}

\begin{document}

In Serverless computing, also known as functions-as-a-service, we are writing short-running, stateless functions that can be triggered by events generated from services or users.

In Faas, a program consists of one or more rules. 

A rule is constructed by the when combinator over a given trigger and action.

So, We have to create a rule to register an action as a new consumer of a trigger.

I defined the rules like this.
    
\begin{verbatim}
    rule(RuleName, TriggerId, ActionId).
\end{verbatim}

When we deploy an application, we had placed Services and Functions. 

I also did Trigger placement in this part. 

\begin{verbatim}
    placeApp(AOp, AId, ...):-
        app(AId, Services),
        placeServices(AOp, Services, ...),
        placeTriggers(AOp, Triggers, ...),
        placeAllFunctions(AOp, ...).
\end{verbatim}

And we are using these triggers in the services. So it is defined in services like this.

    service(SId, Trigger, Program, HWReqs, PReqs, GeoReqList, TimeUnits).


I defined trigger in our application like this.

\begin{verbatim}
    trigger(TId, Prog, Rule).
\end{verbatim}

Here, triggers are uniquely identified by a TId. This trigger will fire Action `Prog`.


To fire a Trigger, I added a fireTrigger function and function placement.

    placeFunctions(AOp, SId, fireTrigger(TId), ...)

And security context of `fireTrigger` function is like this.

\begin{verbatim}
	ctxFire(AOp, fireTrigger(TId), L) :-
        trigger(TId, FId, _),
        func(FId, Args, _, _, _),
        labelF(AOp, Args, L). 
\end{verbatim}

\end{document}